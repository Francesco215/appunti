%\documentclass[11pt,fleqn]{book}%
%\documentclass{exam}%
\documentclass[11pt,a4paper]{article}
\date{Febbraio 2018}
\usepackage[italian]{babel}
\usepackage[T1]{fontenc}
\usepackage{graphicx}
\title{Algebra continua}
\author{Francesco Sacco}
\usepackage[utf8x]{inputenc}
\usepackage{amsmath}
\usepackage{amsthm}
\usepackage{hyperref}

\theoremstyle{definition}
\newtheorem{definizione}{Definizione}

\theoremstyle{plain}
\newtheorem{teorema}{Teorema}

\theoremstyle{plain}
\newtheorem{esempio}{Esempio}


\begin{document}
	\maketitle
	Sono abbastanza convinto di specare tempo, e che forse era meglio se studiavo, ma non avrei studiato comunque quindi faccio questo pdf.\newline
	Ovviamente non sarò formale, quindi Ischia non ci scassare la minchia, e farò parecchi errori grammaticali, quindi Silvio non ci scassare la minchia.\newline
	Un ringraziamento speciale va a Tiziano Amato che mi ha chiesto di metterlo da qualche parte nel pdf, quindi lo metto qua.\newline

	Buona lettura.
	

	\section{Dalle matrici discrete a quelle continue}
		Supponiamo di prendere una matrice $M$ $ n \times m$, adesso rendiamo le cellette della matrice sempre più strette mandando $n$ a infinito.
		\begin{equation}
			\begin{vmatrix}
				a_{1,1} & \dots & a_{1,n} \\
				\vdots	&	\ddots&	\vdots	\\
				a_{n,1} & \dots & a_{n,n} 
			\end{vmatrix}
			\longrightarrow
			\begin{vmatrix}
				&\quad & & \\
				&\ &\quad \cdot \, (x,y) & \\
				& & & \\
			\end{vmatrix}
		\end{equation}
		Così facendo non è più possibile indicare un'elemento della matrice con una coppia di numeri interi, ma è possibile indicarli con una coppia di numeri reali $(x,y)$, di  conseguenza la matrice non è più una funzione $M:\{1,\dots,n\}\times\{1,\dots,m\} \rightarrow \rm I\!R$, ma $M:[a,b]\times [c,d]\rightarrow \rm I\!R$.\newline
		Usando lo stesso ragionamento con i vettori abbiamo che la loro versione con $n\rightarrow +\infty$ siano funzioni $v:[a,b]\rightarrow \rm I\!R$ 
	

	\section{Prodotto caso discreto}
		In algebra lineare un prodotto tra una matrice $M$ e un vettore $v$ non è altro che un vettore $w$ che ha come componente $i$-esima il prodotto scalare canonico tra il vettore $v$ e la $i$-esima riga di $M$, quindi vediamo prima di generalizzare il prodotto scalare.\newline
		Possiamo definire il prodotto scalare in questo modo: prendiamo questa funzione $P:V\times V\rightarrow V$ tale che 
		\begin{equation}
			v=
			\begin{vmatrix}
				v_1\\
				\vdots \\
				v_n
			\end{vmatrix}
			\quad w=
			\begin{vmatrix}
				w_1\\
				\vdots \\
				w_n
			\end{vmatrix}
			\quad P(v,w)=
			\begin{vmatrix}
				v_1 w_1\\
				\vdots \\
				v_n w_n
			\end{vmatrix}
		\end{equation}
		Una volta aver definito questa apparentemente inutile funzione ne definisco un'altra. Prendi un vettore $v=(v_1,\dots,v_n)$ e immagina che ogni componente uscisse dal foglio con un'altezza pari al suo valore (attento agli occhi).\newline
		Adesso se lo guardi di lato dovresti vedere un'istrogramma, nell'immagine che se avrò voglia metterò si capisce benissimo quello che intendo.\newline
		Definisco $A(v)$ l'area di quell'istogramma mostrato dalla fantomatica immagine, è facile verificare che $A(P(v,w))=<v,w>$\footnote{supponendo che alla base dell'istogramma ci sia un quadrato di area unitaria}.

	\section{Prodotto caso continuo}
		Adesso applichiamo lo stesso ragionamento generalizzandolo con le funzioni (che come abbiamo detto sono vettori continui), in questo caso la funzione $P(v(x),w(x))=v(x)w(x)$ e l'area è l'integrale nel dominio di definizione di $v$ e $w$, dunque
		\begin{equation}
			<v(x),w(x)>=\int_a^b v(x)w(x)dx
		\end{equation}
		ovviamente questo soddisfa le proprietà del prodotto scalare, non perchè lo so dimostrare, ma perchè lo so per sentito dire.\newline
		Filamente possiamo definire il prodotto matrice per vettore. Come abbiamo detto prima un prodotto tra una matrice $M$ e un vettore $v$ non è altro che un vettore $w$ che ha come componente $i$-esima il prodotto scalare canonico tra il vettore $v$ e la $i$-esima riga di $M$, quindi il prototto $w$ tra $M$ e $v$ è
		\begin{equation}
			w(y)=\int_a^b M(x,y)v(x) dx
		\end{equation}
		da qui è facile verificare che rispetta tutte le stesse proprietà che ha il prodotto matrice-vettore standard, infatti siano $M, N, O$ matrici continue, $v,w$ vettori, e $\lambda$ uno scalare, allora 
		\begin{equation}
			\lambda Mv= M(\lambda v),\quad M\circ (N \circ O)= (M \circ N) \circ O,
		\end{equation}
		\[
			Mv + Nv=  (M+N)v,\quad Mv+Mw=M(v+w)
		\]
		
		Lascio al lettore l'esercizio di dimostrare queste poprietà, trovare la matrice identità e la formula per prodotto tra matrici\footnote{comunque se scrolli dovresti trovarle la soluzione}
	

	\section{Combinazioni lineari}
		In algebra lineare sappiamo che se $w\in V$ e ${v_1,\dots,v_n}$ è una base di $V$ abbiamo che esistono dei coefficenti $a_1,\dots,a_n$ tali che $w=a_1 v_1+\dots +a_n v_n$, scritto in forma vettoriale
		\begin{equation}
			\begin{vmatrix}
				w_1\\
				\vdots\\
				w_n
			\end{vmatrix}
			=
			\begin{vmatrix}
				& & \\
				v_1 & \dots & v_n\\
				& & 
			\end{vmatrix}
			\begin{vmatrix}
				a_1\\
				\vdots\\
				a_n
			\end{vmatrix}
		\end{equation}
		dove la matrice nel mezzo è la matrice che ha come colonne i vettori $v_1,\dots,v_n$, quindi un modo per calcolare i coefficenti $a_1,\dots,a_n$ è quello di calcolare l'inversa della matrice\footnote{che esiste perchè le righe sono lienarmente indipendenti} e moltiplicare membro a membro.\newline

		La stessa cosa si può fare con le matrici continue, ma prima parliamo di matrici inverse.\newline
		Sarei tentato di definire la matrice $M^{-1}$ l'inversa di $M$ se $Id(x,y)=\delta (y-x)=\int_a^b M^{-1}(x,t) M(t,y) dt$, ma per ora la definisco così:\newline
		\begin{definizione}[Matrice inversa]
			Sia $M:[a,b]\times [a,b]\rightarrow \rm I\!R$, e $N$ pure (se c'hai voglia te lo generalizzi tu), allora $N$ è l'inversa di $M$ se
			\begin{equation}
				v=NMv
			\end{equation}
		\end{definizione}
		Adesso che sappiamo cos'è l'inversa possiamo usare lo stesso ragionamento della scorsa sezione: Supponiamo che $\{M(x,y_0)\,|\, y_0\in \rm I\!R \}$ sia una base di uno spazio vettoriale $V$, allora per ogni $v\in V$ è possibile trovare una combinazione lineare dei vettori di base che sia uguale a $v$ stesso, cioè
		\begin{equation}
		\label{combinazione lineare matrice}
			\begin{vmatrix}
				\\
				v(x)\\
				\quad
			\end{vmatrix}
			=
			\begin{vmatrix}
				\, & & \, \\
				& M(x,y) & \\
				& & 
			\end{vmatrix}
			\begin{vmatrix}
				\\
				a(y)\\
				\quad
			\end{vmatrix}
		\end{equation}
		Di conseguenza se si vuole trovare $a(y)$ basta moltiplicare per l'inversa entrambi i lati.

	\section{Serie di Furiè \protect\footnote{se sai come funziona puoi saltarlo, e comunque fa più confondere che altro}}
		Chi ha un'occhio attento avrà già notato che la trasformata di fourier è un prodotto matrice-vettore, ma prima di spiegare per bene il motivo è meglio cominciare dalla sua sorellina minore: la serie di fourier\newline 

		Partiamo definendo un prodotto hermitiano che ci faccia comodo
		\begin{equation}
			<f(x),g(x)>=\frac{2}{T}\int_{-\frac{T}{2}}^\frac{T}{2} \overline{f(x)}g(x)dx
		\end{equation}
		se prendiamo questo insieme di vettori $E_T=\{e^{i2\pi n/ T} \, | \, n \in \rm I\!\! N\}$ abbiamo che sono vettori ortonormali rispetto a questo prodotto scalare, quindi sono perfetti per usarli come base dello spazio vettoriale $C\big[-\frac{T}{2},\frac{T}{2}\big]$\footnote{a dire il vero bisognerebbe dimostrare che $Span\{E_T\}$ sia effettivamente $C\big[-\frac{T}{2},\frac{T}{2}\big]$}, o per le funzioni che oscillano con un periodo $T$.\newline
		Tutto ciò ci porta alla celeberrima serie di fouriè, dove $\widehat{e_n}=e^{i2\pi n/ T}$
		\begin{equation}
		\label{serieFuriè}
			f(x)=Re\Bigg[\sum_{n=0}^{+\infty} <\widehat{e_n},f(x)>\widehat{e_n}\Bigg]
		\end{equation}
		Scritto in forma vettoriale assume una forma strana, è una matrice semi continua, che sarebbe $M:[a,b]\times \{1,\dots,n\}\rightarrow \rm I\!R$
		\begin{equation}
			\begin{vmatrix}
				\,\\
				\,\\
				f(x)\\
				\,\\
				\,
			\end{vmatrix}
			=Re
			\begin{vmatrix}
				\, & & &\\
				\, & & &\\
				\widehat{e_0}& \dots & \widehat{e_n}&\dots\\
				\, & & &\\
				\, & & &
			\end{vmatrix}
			\begin{vmatrix}
				<f(x),\widehat{e_0}>\\
				\vdots\\
				<f(x),\widehat{e_n}>\\
				\vdots
			\end{vmatrix}
		\end{equation}
		So che è un pò complicato, ma se avete dubbi chiedetemi di persona\footnote{il mio numero è +39 345 7011798}


	\section{Trasformata di Furiè}
		Ora che abbiamo reso incomprenzibile la serie, facciamo lo stesso lavoro con la trasformata.\newline \newline
		La trasformata di fourier, può essere vista come una serie di forurier con il periodo che tende a infinito, ora però c'è il problema di scegliere una base.\newline
		Non essendoci stavolta un valido motivo per escludere le funzioni periodiche che hanno la frequenza che non siano un multiplo di $1/T$ possiamo prenderci la libertà di prenderle tutte. Di conseguenza segliamo come base $B=\{F(x,y_0)=e^{ixy_0}\,|\,y_0 \in \rm I\!R \}$.\newline
		Come abbiamo visto nell'equazione \ref{combinazione lineare matrice} possiamo scrivere una generica funzione in questo modo
		\begin{equation}
			\begin{vmatrix}
				\\
				f(x)\\
				\quad
			\end{vmatrix}
			=
			\begin{vmatrix}
				\, & & \, \\
				& F(x,y) & \\
				& & 
			\end{vmatrix}
			\begin{vmatrix}
				\\
				\hat{f}(y)\\
				\quad
			\end{vmatrix}
		\end{equation}
		La quale scritta in forma esplicita diventa la trasformata di Fourier
		\begin{equation}
		f(x)=\int_{-\infty}^{+\infty} e^{ixy}\hat{f}(y) dy
		\end{equation}
		Da qui si evince cosa rappresenta fisicamente $\hat{f}(y)$: essa è la funzione che indica i coefficienti da mettere davanti i vettori di base per ottenere il vettore desiderato.

	\section{Matrice associata a un'applicazione lineare}
		Per costruire una matrice associata a un'applicazione lineare il procedimento è pressocchè identico a quello adottato in algebra lineare:\newline
		Sia $L:V\rightarrow W$ un'applicazione lineare tra due spazii vettoriali e sia $B=\{v_{y}(x)|y\in \rm I\!R\}$ una base di $V$. Allora la matrice associata all'applicazione lineare sarà $L(x,y)=L(v_{y}(x))$.\newline
		Detto in parole povere ho messo in ogni colonna dove va a finire il vettore di base associato.\newline

	\section{Cambiamento di base}
		Trovare la matrice associata all'applicazione lineare ci dà informazioni su come trasformano i vettori di base, ma non ci dice molto su come trasforma una funzione generica.\newline
		Sia $B=\{v_{y}(x)|y\in \rm I\!R\}$ una base di uno spazio vettoriare $V$ denso in $\rm I\!L$, sia $M$ la matrice assiociata all'applicazione lineare rispetto alla base $B$ e sia $N(x,y)=v_{y}(x)$ la matrice di cambiamento di base, allora la matrice associata all'applicazione lineare $L$ in $V$ sarà
		\begin{equation}
			L=N^{-1}\circ M \circ N
		\end{equation}
		Che è praticamente identica alla sua controparte discreta di algebra lineare

	\section{Proprietà $\delta$ di Dirac}
		Prima di ricavarci la matrice associata alla derivata occorre sapere un paio di proprietà sulla $\delta$ di dirac.\newline
		La prima è sapere qual'è la sua derivata, che nonostante sia un pò un'abuso di notazione fa il suo sporco lavoro. Partiamo dall'osservare che $x\delta (x)=0$ per ogni $x$, adesso calcoliamoci la sua derivata.
		\[
			(x\delta (x))'=0= x\delta'(x) + \delta(x)
		\]
		risolvendo per $\delta'(x)$ si ottiene che 
		\begin{equation}
			\delta'(x)=-\frac{\delta(x)}{x}
		\end{equation}
		La seconda proprietà non la dimostro, essa afferma che
		\begin{equation}
			\delta(x-y)=\frac{1}{2\pi} \int_{-\infty}^{+\infty} e^{it(x-y)} dt
		\end{equation}

	\section{Matrice associata alla derivata}
		A questo punto possiamo trovare la matrice associata alla più famosa delle applicazioni lineare: la derivata\footnote{Se ti senti particolarmente intelligente puoi provare a trovare la matrice associata all'integrale}.\newline
		Cominciamo mettendoci nella base $B=\{e^{ixy_0}\,|\,y_0 \in \rm I\!R \}$ delle esponenziali complesse, la matrice associata alla derivazione rispetto a $x$ $\widehat{D}$ in quella base è 
		\begin{equation}
			\widehat{D}(x,y)=-iy\delta (x-y)=-ix\delta(x-y)=-i\sqrt{|xy|}\delta(x-y)
		\end{equation}
		Mentre la matrice di cambiamento di base è $F(x,y)=\frac{e^{-iyx}}{\sqrt{2\pi}}$, e la sua inversa è $F^{-1}(x,y)=\frac{e^{iyx}}{\sqrt{2\pi}}$, per avere l'applicazione derivata $D$ rispetto alla base canonica bisognerà comporre le matrici
		\begin{equation}
			D=F^{-1}\circ \widehat{D}\circ F
		\end{equation}
		Essendo la composizione di matrici associativa, conviene calcolarci prima $\widehat{D}F$
		\[
			(\widehat{D}\circ F)(x,y)=\frac{1}{\sqrt{2\pi}}\int_{-\infty}^{+\infty}
			-iy\delta(y-t)e^{-itx}dt=-\frac{iy}{\sqrt{2\pi}}e^{-iyx}
		\]
		adesso basta moltiplicare la nuova matrice per $F^{-1}$
		\[
			(F^{-1}\circ \widehat{D}\circ F)=
			\frac{1}{\sqrt{2\pi}}\int_{-\infty}^{+\infty}-\frac{it}{\sqrt{2\pi}}e^{itx}e^{-ity}dt=
			-\frac{i}{2\pi}\int_{-\infty}^{+\infty}te^{it(x-y)}dt
		\]
		integrando per parti si ottine
		\[
			-\frac{it}{2\pi}(2\pi\delta(x-y))\bigg\rvert_{-\infty}^{+\infty}+
			\frac{i}{2\pi}\int_{-\infty}^{+\infty}\frac{e^{it(x-y)}}{i(x-y)}dt=
			\frac{1}{2\pi(x-y)}\int_{-\infty}^{+\infty}e^{it(x-y)}dt=\frac{\delta(x-y)}{x-y}
		\]
		E così abbiamo la matrice associata alla derivata nella base canonica delle funzioni
		\begin{equation}
			D(x,y)=\frac{\delta(x-y)}{x-y}
		\end{equation}
		Adesso dimostriamo che effettivamente funziona
		\[
			\int_{-\infty}^{+\infty}\frac{\delta(x-y)}{x-y}f(y)dy=-\delta(x-y)f(y)\big\rvert_{-\infty}^{+\infty}+\int_{-\infty}^{+\infty}\delta(x-y)f'(y)dy=f'(x)
		\]
	\section{Matrice associata alla derivata n-esima}
		Ora che avete visto la matrice associata alla derivata, vediamo se si può fare la stessa cosa per la derivata n-esima.
		Nella base delle esponenziali complesse la matrice che cerchiamo è 
		\begin{equation}
			\widehat{D_n}(x,y)=(-iy)^n\delta(x-y)
		\end{equation}
		A questo punto effettuiamo il cambiamento di base di quest'altra matrice
		\[
			(\widehat{D_n}\circ F)(x,y)=
			\frac{1}{\sqrt{2\pi}}\int_{-\infty}^{+\infty}(-iy)^n\delta(y-t)e^{-itx}dt=
			\frac{(-iy)^n}{\sqrt{2\pi}}e^{-iyx}
		\]
		\[
			D_n=(F^{-1}\circ \widehat{D_n}\circ F)=
			\frac{1}{\sqrt{2\pi}}\int_{-\infty}^{+\infty}\frac{(-it)^n}{\sqrt{2\pi}}e^{itx}e^{-ity}dt=
			\frac{(-i)^n}{2\pi}\int_{-\infty}^{+\infty}t^n e^{it(x-y)}dt
		\]
		Integrando per parti si ottiene che
		\[
			\frac{(-it)^n}{2\pi}(2\pi\delta(x-y))\bigg\rvert_{-\infty}^{+\infty}-
			\frac{(-i)^n}{2\pi}\int_{-\infty}^{+\infty}nt^{n-1}\frac{e^{it(x-y)}}{i(x-y)}dt=
			\frac{n(-i)^{n-1}}{2\pi(x-y)}\int_{-\infty}^{+\infty}t^{n-1}e^{it(x-y)}dt=	
		\]
		Riscrivendo l'ultima equazione in termini di $D_{n-1}$ e applicandola ricorsivamente si ottiene la formula della marice associata alla derivata n-esima\footnote{Sinceramente non è che mi convinci molto questo risultato, ma per ora farò finta che funzioni. Se riuscite a trovarmi un contr'esempio, o una formula "più corrretta" mi fate un favore}
		\begin{equation}
			D_{n}=\frac{n}{x-y}D_{n-1}=n!\frac{\delta(x-y)}{(x-y)^n}
		\end{equation}	

	\section{Risoluzione equazioni differenziali linearii a coefficenti costanti}
		Se si vuole trovare la matrice associata ad un'equazione differenziale lineare basta prenderne una a caso $a_nf^{n}(x)+\dots +a_0f(x)=g(x)$ e scriverla in forma matriciale $a_nD_nf+\dots+a_0D_0f=g$.\newline
		Se definiamo $L=a_nD_n+\dots+a_0D_0$ si ottiene che $L$ è la matrice associata alla nostra equazione differenziale $Lf=g$.\newline
		Di conseguenza abbiamo davanti a noi un problema che è deltutto analogo alla risoluzione di un sistema di equazioni lineri in algebra lineare $Mv=w$.\newline
		Quindi la soluzione più generica sarebbe il nucleo $N$ di $L$ sommato alla soluzione dell'inversa ristretta all'immagine $I$ di $L^{-1}_{|I}g=f$.\newline

		A prima lettura sembra un problema veramente complicato, tuttavia bisogna ricordarsi che nel caso in cui si ci trova in una base di autovettori di $D_i$, $L$ risulta diagonale, e la cosa bella delle matrici diagonali è che il nucleo sono gli zeri sulla diagonale e che l'iversa è l'inverso degli elementi sulla diagonale.\newline
		Visto che le esponenziali sono una base\footnote{A dire il vero dovrei dimostrarlo, ma per ora fidati, comunque se ti interessa perchè, vatti a studiare la trasformata di Laplace. In ogni caso lo stesso identico risultato si sarebbe ottenuto con le esponenziali complesse, ma sarebbe stato più fastidioso da ricavare a causa dei numeri complessi} di autovettori rispetto alla derivata conviene risolvere il probema rispetto ad essa.\newline

		Rispetto alla base $B=\{e^{yx}|y \in \rm I\!R\}$ $D_n=y^n\delta(x-y)$, di conseguenza $L(x,y)=\delta(x-y)(a_ny^n+\dots+a_0)$, quindi gli zeri lungo la diagonale $y_0,\dots,y_n$ sono le soluzioni di $P(y)=a_ny^n+\dots+a_0$, inoltre per il teorema fondamentale dell'algebra si ha che $dim(N)=n$. Il nucleo sarà quindi nella forma $N=span\{e^{y_0x},\dots,e^{y_nx}\}$\footnote{Se la tua domanda è:"Ma tu vu futtiri a mia?! Che succede se delle radici sono coincidenti?" ti posso soltanto dire che ci sto lavorando}.\newline

		Per trovare l'omogenea $f_o$ bisognerà risolvere $f_o=L^{-1}_{|I}g$, ma per ora mi secca scrivere come si fa.

	\section{Prodotto Tensoriale\protect\footnote{Per capire questa parte bisogna sapere almeno le basi sui tensori e il prodotto tensoriale, consiglio di leggere \href{url}{https://jeremykun.com/2014/01/17/how-to-conquer-tensorphobia}}}
		Prendiamo due vettori continui $v$ e $w$ il loro prodotto tensoriale $v\otimes w$ è scrivibile come combinazione dei prodotti tensoriali degli elementi della base $B=\{B(x,y_0)|y_0\in \rm I\!R\}$ dello spazio vettoriale $V$, quindi esprimiamo $v$ e $w$ in termini di combinazione linare dei vettori di base, cioè $v(x)=\int B(x,y)a(y)dy$ e $w(x')=\int B(x',y')b(y')dy'$
		\[
			v(x)\otimes w(x')=\bigg(\int B(x,y)a(y)dy \bigg)\otimes \bigg(\int B(x',y')b(y')dy'\bigg)=
		\]
		\[
			=\int \bigg[B(x,y)\otimes\bigg(\int B(x',y')b(y')dy'\bigg)\bigg]a(y)dy=
		\]
		\[
			=\int \int \bigg(B(x,y)\otimes B(x',y')\bigg)a(y)b(y')dydy'
		\]
		A questo punto basta definire $T(y,y')=B(x,y)\otimes B(x',y')$\footnote{Aspetta! ma ti sei mangiato 2 indici!\\ No. Nel prodotto tensoriale "classico" $e_i\otimes f_j=k$ è comunque uno scalare $k$, nonostante possa sembrare a prima occhiata che $k$ dipenda sia da $i$ che da $j$, infatti $e_i$ serve solo per indicare la componente $i$-esima di $e$} il prodotto tensoriale tra gli elementi di base $B(x,y)$ e $B(x,y')$ e l'equazione di sopra diventa
		\begin{equation}
			v(x)\otimes w(x')=\int \int T(y,y')a(y)b(y')dydy'
		\end{equation}
		Adesso cerchiamo di generalizzare un pò quello che abbiamo fatto fin'ora: prendiamo il prodotto tensoriale $\otimes:\underbrace{V^*\times\dots\times V^*}_{p \textrm{ volte}}\times\underbrace{V\times\dots\times V}_{q \textrm{ volte}}\times\rightarrow\rm I\!R\ $.\newline
		Questi tensori sono detti di tipo $(p,q)$


	\section{Tensori fatti male\protect\footnote{Questa parte presenterà diverse lacune e imperfezioni in quanto non ho studiato per bene i tensori ed è scritta pure male}}
		Un tensore $T$ di tipo $(n,m)$ può essere visto come una specie di matrice con $n+m$ indici. La versione continua di un tensore è quindi qualcosa del genere $T^{(x_1,\dots,x_n)}_{(y_1,\dots,y_m)}$, dove $(x_1,\dots,x_n)$ e $(y_1,\dots,y_m)$ sono dei vettori di $\rm I\!R^n$ e $\rm I\!R^m$ ed ad ogni coppia di punti è associato uno scalare.\newline
		Se facciamo il limite al continuo degli indici otteniamo che il tensore diventa della forma $T^{f}_{g}$, dove $f$ e $g$ sono due vettori continui.

	\section{Funzionali}
		Adesso concentriamoci sui Tensori con indici continui che hanno solo una funzione negli indici $T_f$, che da ora in poi scriverò come $T[f(x)]$. Essi sono funzioni che vanno dallo spazio delle funzioni in $\rm I\!R$. Questi oggetti in matematica sono definiti come Funzionali.\newline
		Esistono diversi tipi di funzionali, ma io mi concentrerò su quelli più facili, cioè quelli tali che
		\begin{equation}
			\lim_{\epsilon 	\to 0}T[f(x)+\epsilon g(x)]-T[f(x)]=0
		\end{equation}
		Che chiamerò funzionali continui\footnote{Probablilmente ce l'hanno una definizione per come si deve, ma io non la so, quindi ci metto questa perchè assomigliano alle funzioni continue di Analisi 1}.


	\section{Derivata funzionale}
		Le analogie tra le funzioni in $n$ variabili e i funzionali sono molte, una di queste è la derivata direzionale.\newline
		Nelle funzioni in più variabili la derivata direzionale è\footnote{ricordati che $||y||_2=1$} $\lim_{\epsilon \to 0}\frac{f(\mathbf{x}+\epsilon\mathbf{y})-f(\mathbf{x})}{\epsilon}$. Nei funzionali invece di prendere $\mathbf{x},\mathbf{y}$ vettori discreti, prendiamo vettori continui, e così facendo si ottene la derivata funzionale
		\begin{equation}
			\frac{dF}{dg}[f(x)]=\lim_{\epsilon \to 0}\frac{F[f(x)+\epsilon g(x)]-F[f(x)]}{\epsilon}
		\end{equation}
		con $||g||_2=1$. Questa equazione rappresenta la derivata di $F$ nella direzione di $g$.\newline
		Se si vuole trovare un punto stazionario di un funzionale $F$ bisogna imporre che $\frac{dF}{dg}=0$ per ogni $g$, che equivarrebe a dire che la derivata deve essere zero in ogni direzione.\newline


	\section{Equazioni di Eulero-Lagrange}
		L'esempio più famoso nel trovare i punti stazionari di un funzionale è l'equazione di Eulero-Lagrange:
		\begin{equation}
			A[x(t)]=\int_{t_a}^{t_b}L(x(t),\dot{x}(t),t)dt
		\end{equation}
		dove $A$ è un funzionale detto Azione che per qualche oscuro motivo dobbiamo minimizare, $L$ è la lagrangiana, $x$ è la posizione, $\dot{x}$ la velocità e $t$ il tempo.
		Per trovare il minimo bisogna imporre che per ogni $y$
		\[
			\frac{dA}{dy}=\lim_{\epsilon \to 0}\frac{A[x(t)+\epsilon y(t)]-A[x(t)]}{\epsilon}=0
		\]
		sviluppo in Taylor il primo membro
		\[
			A[x(t)+\epsilon y(t)]=\int_{t_a}^{t_b}L(x+\epsilon y,\dot{x}+\epsilon \dot{y},t)=A[x(t)]+\int_{t_a}^{t_b}\frac{\partial{L}}{\partial{x}}\epsilon y+\frac{\partial{L}}{\partial{\dot{x}}}\epsilon \dot{y}dt
		\]
		quindi
		\[
			\frac{dA}{dy}=\int_{t_a}^{t_b}\frac{\partial{L}}{\partial{x}}y+\frac{\partial{L}}{\partial{\dot{x}}}\dot{y}dt
		\]
		integrando per parti il secondo termine dell'integrale
		\[
			\int_{t_a}^{t_b}\frac{\partial{L}}{\partial{\dot{x}}}\dot{y}dt=y\frac{\partial{L}}{\partial{\dot{x}}}\bigg\rvert_{t_a}^{t_b}-
			\int_{t_a}^{t_b}\frac{d}{dt}\bigg(\frac{\partial{L}}{\partial{\dot{x}}} \bigg)ydt
		\]
		Adesso per semplicità asumiamo che i punti estremali della traiettoria sono fissati, quindi $y(t_a)=y(t_b)=0$, quindi rimettendo tutto nell'equazione precedente si ha che
		\begin{equation}
			\frac{dA}{dy}=\int_{t_a}^{t_b}\bigg[\frac{\partial{L}}{\partial{x}}-\frac{d}{dt}\bigg(\frac{\partial{L}}{\partial{\dot{x}}} \bigg)\bigg]ydt
		\end{equation}
		che è uguale a zero per ogni $y(t)$ se e solo se 
		\begin{equation}
			\frac{\partial{L}}{\partial{x}}=\frac{d}{dt}\bigg(\frac{\partial{L}}{\partial{\dot{x}}} \bigg)
		\end{equation}
\end{document}