%\documentclass[11pt,fleqn]{book}%
%\documentclass{exam}%
\documentclass[11pt,a4paper]{article}
\date{marzo 2018}
\usepackage[italian]{babel}
\usepackage[T1]{fontenc}
\usepackage{graphicx}
\title{Matrici continue}
\author{Francesco Sacco}
\usepackage[utf8x]{inputenc}
\usepackage{amsmath}
\usepackage{amsthm}




\begin{document}
	\maketitle
		Sono abbastanza convinto di specare tempo, e che forse era meglio se studiavo, ma non avrei studiato comunque quindi faccio questo pdf.\newline
		Ovviamente non sarò formale, quindi Ischia non ci scassare la minchia, e farò parecchi errori grammaticali,quindi Silvio non ci scassare la minchia.
		Buona lettura.
		\section{Dalle matrici discrete a quelle continue}
			Supponiamo di prendere una matrice $M$ $ n \times n$, adesso rendiamo le cellette della matrice sempre più strette mandando $n$ a infinito.
			\begin{equation}
				\begin{vmatrix}
					a_{1,1} & \dots & a_{1,n} \\
					\vdots	&	\ddots&	\vdots	\\
					a_{n,1} & \dots & a_{n,n} 
				\end{vmatrix}
				\longrightarrow
				\begin{vmatrix}
					& & & \\
					& & & \\
					& & (x,y) & \\
					& & & \\
				\end{vmatrix}
			\end{equation}
			Così facendo non è più possibile indicare un'elemento della matrice con una coppia di numeri interi, ma è possibile indicarli con una coppia di numeri reali $(x,y)$, di  conseguenza la matrice non è più una funzione $M:n\times n \rightarrow \rm I\!R$, ma $M:[a,b]\times [c,d]\rightarrow \rm I\!R$.\newline
			Usando lo stesso ragionamento con i vettori abbiamo che la loro versione con $n\rightarrow +\infty$ siano funzioni $v:[a,b]\rightarrow \rm I\!R$ 
		\section{Prodotto}
			In algebra lineare un prodotto tra una matrice $M$ e un vettore $v$ non è altro che un vettore $w$ che ha come componente $i$-esima il prodotto scalare canonico tra il vettore $v$ e la $i$-esima riga di $M$, quindi vediamo prima di generalizzare il prodotto scalare.\newline
			Possiamo definire il prodotto scalare in questo modo: prendiamo questa funzione $P:V\times V\rightarrow V$ tale che 
			\begin{equation}
				v=
				\begin{vmatrix}
					v_1\\
					\vdots \\
					v_n
				\end{vmatrix}
				\quad w=
				\begin{vmatrix}
					w_1\\
					\vdots \\
					w_n
				\end{vmatrix}
				\quad P(v,w)=
				\begin{vmatrix}
					v_1 w_1\\
					\vdots \\
					v_n w_n
				\end{vmatrix}
			\end{equation}
			Una volta aver definito questa apparentemente inutile funzione ne definisco un'altra.
			prendi un vettore $v=(v_1,\dots,v_n)$ e immagina che ogni componente uscisse dal foglio con un'altezza pari al valore che ha (attento agli occhi).\newline
			Adesso se lo guardi di lato dovresti vedere un'istrogramma, nell'immagine che se avrò voglia metterò si capisce benissimo quello che intendo.\newline
			Definisco $A(v)$ l'area di quell'istogramma mostrato dalla fantomatica immagine, è facile verificare che $A(P(v,w))=<v,w>$ (supponendo che alla base dell'istogramma ci sia un quadrato di area unitaria).\newline
			Adesso applichiamo lo stesso ragionamento possiamo generalizzarlo con le funzioni (che come abbiamo detto sono vettori continui), in questo caso la funzione $P(v(x),w(x))=v(x)w(x)$ e l'area è l'integrale nel dominio di definizione di $v$ e $w$, dunque
			\begin{equation}
			<v(x),w(x)>=\int_a^b v(x)w(x)dx
			\end{equation}
			ovviamente questo soddisfa le proprietà del prodotto scalare, non perchè lo so dimostrare, ma perchè lo so per sentito dire.\newline
			Filamente possiamo definire il prodotto matrice per vettore. Come abbiamo detto prima un prodotto tra una matrice $M$ e un vettore $v$ non è altro che un vettore $w$ che ha come componente $i$-esima il prodotto scalare canonico tra il vettore $v$ e la $i$-esima riga di $M$, quindi il prototto $w$ tra $M$ e $v$ è
			\begin{equation}
				w(y)=\int_a^b M(x,y)v(x) dx
			\end{equation}
			da qui è facile verificare che rispetta tutte le stesse proprietà che ha il prodotto matrice-vettore standard.\newline
			Lascio al lettore l'esercizio di trovare la matrice identità e di definire il prodotto tra matrici (se scrolli dovresti trovarle la soluzione) 
		\section{cose con lo span parte discreta}
			In algebra lineare sappiamo che se $w\in V$ e ${v_1,\dots,v_n}$ è una base di $V$ abbiamo che esistono dei coefficenti $a_1,\dots,a_n$ tali che $w=a_1 v_1+\dots +a_n v_n$, scritto in forma vettoriale
			\begin{equation}
				\begin{vmatrix}
					w_1\\
					\vdots\\
					w_n
				\end{vmatrix}
				=
				\begin{vmatrix}
					& & \\
					v_1 & \dots & v_n\\
					& & 
				\end{vmatrix}
				\begin{vmatrix}
					a_1\\
					\vdots\\
					a_n
				\end{vmatrix}
			\end{equation}
			dove la matrice nel mezzo è la matrice che ha come colonne i vettori $v_1,\dots,v_n$, quindi un modo per calcolare i coefficenti $a_1,\dots,a_n$ è quello di calcolare l'inverso della matrice e moltiplicare membro a membro.\newline
		\section{cose con lo span parte continua}
			La stessa cosa si può fare con le matrici continue, ma prima parliamo di matrici inverse.
			Sarei tentato di definire la matrice $M^{-1}$ l'inversa di $M$ se $Id(x,y)=\delta (x,y)=\int_a^b M^{-1}(x,t) M(t,y) dt$, ma per ora la definisco così:\newline
			\newline
				Sia $M:[a,b]\time [a,b]\rightarrow \rm I\!R$, e $N$ pure (se c'hai voglia te lo generalizzi tu), allora $N$ è l'inversa di $M$ se
				\begin{equation}
					v=NMv
				\end{equation}
		

\end{document}