\documentclass[10pt,a4paper]{article}
\usepackage{verbatim}
\usepackage{subcaption}
\usepackage{karnaugh-map}
\usepackage[utf8]{inputenc}
\usepackage[italian]{babel}
\usepackage{amsmath}
\usepackage{amsfonts}
\usepackage{amssymb}
\usepackage{graphicx}
\usepackage[left=2cm,right=2cm,top=2cm,bottom=2cm]{geometry}
\newcommand{\rem}[1]{[\emph{#1}]}
\newcommand{\exn}{\phantom{xxx}}
\newcommand{\limx}{\lim_{dx\to 0}}

\author{Francesco Sacco}
\title{Proof that the derivative of $e^x$ is itself}
\begin{document}
\date{24 Aprile 2019}
\maketitle
In this document i'm going to use just the definition of derivative and basic algebraic rules to prove that the solution to $f'(x)=\tau f(x)$ is $e^{\tau x}$.
\section*{Proof}
	let's start by writing $f'(x)$ in terms of the limit.
	\begin{equation}
		f'(x)=\limx f'(x-dx)=\lim_{dx\to 0}\frac{f(x)-f(x-dx)}{dx}=\limx\tau f(x-dx)
	\end{equation}
	I've already used the fact that the derivative must be continuos\footnote{Look at the appendix to see why that must be true, but for now i think you can survive by assuming it to be true}.\newline
	Now we can isolate $f(x)$ by bringing on the right side everything else.
	\begin{equation}
		f(x)=\limx f(x-dx)(\tau dx+1)
	\end{equation}
	that means that
	\[
		\limx f(x-dx)=\limx f(x-2dx)(\tau dx+1) \rightarrow f(x)=\limx f(x-2dx)(\tau dx+1)^2
	\]
	and
	\[
		\limx f(x-2dx)=\limx f(x-3dx)(\tau dx+1) \rightarrow f(x)=\limx f(x-3dx)(\tau dx+1)^3
	\]
	continuing this thing $n$ times we get
	\begin{equation}
		f(x)=\limx f(x-ndx)(\tau dx+1)^n
	\end{equation}
	that is true for every $n$\footnote{To be fear $n$ should be an integer, but we can choose either $dx$ to be a integer divisor of $x$ or we can do some trickery whidt some $\epsilon$s and $\delta$s. For now i don't think this is really important}, if we choose to set $n=\frac x{dx}$ we get that
	\begin{equation}
		f(x)=\limx f(0)(\tau dx+1)^\frac x{dx}
	\end{equation}
	if we multiply and divide the exponent by $\tau$ e define $k=\tau dx$ we get
	\begin{equation}
		f(x)=f(0)\big[\lim_{k\to 0}(k+1)^{1/k}\big]^{\tau x}
	\end{equation}
	if we define $e$ the number to witch the limit $\lim_{k\to 0}(k+1)^{1/k}$ converges we can write\footnote{How can we be so sure that the limit converges? Look at the appendix.}
	\begin{equation}
		f(x)=f(0)e^{\tau x}
	\end{equation}


\section*{Appendix}
	Scrivi quelle cose


\end{document}