\documentclass{exam}
\date{Dicembre 2016}
\usepackage[italian]{babel}
\usepackage[T1]{fontenc}
\title{Appunti di Analisi Matematica}
\author{Francesco Sacco}
\begin{document}





\section{Limiti}
  \subsection{Definizioni}
    \begin{enumerate}
      \item 
        sia $f(x) \in D\subset R$ e sia $x_{0}$ un punto di accumulazione di $D$. diremo che
        \begin{displaymath}
          (\lim_{x \to x_{0}}=L\Leftrightarrow \forall \varepsilon , \exists \delta > 0 :
          \forall x \in D \land 0<|x-x_{0}|< \delta) \Rightarrow |f(x)-L|<\varepsilon
        \end{displaymath}
      \item
        $\textrm{sia } f(x) \in D \subset \mathbf{R} \land x_{0}\in f(x) $
        \begin{displaymath}
          \textrm{se } \forall M \in \mathbf{R} \exists \delta >0: \forall x \subset D , 
          0<|x-x_{0}|<\delta \Rightarrow f(x)>M
          \textrm{ allora} \lim_{x \to x_{0}}f(x)=\infty
        \end{displaymath}
      \item 
        sia $f(x) \in D \subset \mathbf{R}$ non limitato superiormente
        \begin{displaymath}
          \lim_{x \to \infty}=L \Leftrightarrow 
          \forall \varepsilon>0 \exists \nu:\forall x\in D,
          x>\nu \Rightarrow |f(x)-L|<\varepsilon
        \end{displaymath}
      \item
        \(\displaystyle
          \textrm{sia } f(x) \in D\subset \mathbf{R}
          \textrm{ se }\lim_{x\to x_{0}}=\infty \textrm{ allora diremo che la funzione }f(x) 
          \textrm{ ha un'asintoto verticale in }x_{0}
        \)
      \item 
        sia $f(x)\in (c, \pm \infty)$ Diremo che la retta di equazione $y=ax+b$
        \'e un asintoto di $f$ se
          \begin{displaymath}
            \lim_{x\to \pm \infty}[f(x)-ax-b]=0
          \end{displaymath}
    \end{enumerate}
  
  
  
  \subsection{Teoremi}
    \begin{enumerate}
      \item siamo $f(x)$ e $g(x)$ due funzioni tali che
        \begin{displaymath}
          \lim_{x\to x_{0}}f(x)=L
        \end{displaymath}
        \begin{displaymath}
          \lim_{x\to x_{0}}g(x)=M
        \end{displaymath}
        \begin{displaymath}
          \textrm{allora}
        \end{displaymath}
        \begin{displaymath}
          \lim_{x \to x_{0}} [f(x)+g(x)]= L+M
        \end{displaymath}
        \begin{displaymath}
          \lim_{x \to x_{0}} f(x)g(x)=LM
        \end{displaymath}
        \begin{displaymath}
          \lim_{x\to x_{0}} \frac{f(x)}{g(x)}=\frac{L}{M}
        \end{displaymath}
      \item{teorema della permanenza del segno}
        \begin{displaymath}
          \textrm{se} \lim_{x\to x_{0}} f(x)=L>0 \Rightarrow \exists \delta >0 : \forall x :
          0<|x-x_{0}|<\delta \textrm{ si ha } f(x)>\frac{M}{2}>0
        \end{displaymath}
      \item{teorema dei 2 carabinieri}
        \begin{displaymath}
          \textrm{siano} f(x),g(x)\textrm{ e } h(x) \in \mathbf{R} \land f(x)<g(x)<h(x) 
          \land \lim_{x \to x_{0}}f(x)=\lim_{x \to x_{0}}h(x)=L
        \end{displaymath}
        \begin{displaymath}
          \textrm{allora }\lim_{x \to x_{0}}g(x)=L
        \end{displaymath}
    \end{enumerate}
  
  
  
  \subsection{Limiti notevoli}
    \begin{center}
      \begin{tabular}{|c|c|}
        \(\displaystyle \lim_{x\to 0}\frac{\textrm{sen}(x)}{x}=1\)&
        \(\displaystyle \lim_{x\to 0}\frac{1-\textrm{cos}(x)}{x}=\frac{1}{2}\)\\
        \(\displaystyle \lim_{x\to 0}\frac{\textrm{tan}(x)}{x}=1\)&
        \(\displaystyle \lim_{x\to 0}\frac{\textrm{arcCos}(x)}{\sqrt[]{1-x}}=\sqrt[]{2}\)\\
        \(\displaystyle \lim_{x\to 0}(1+x)^\frac{1}{x}=e\)&
        \(\displaystyle \lim_{x\to 0}(1-x)^\frac{1}{x}=\frac{1}{e}\)\\
        \(\displaystyle \lim_{x\to \infty}\Big(1+\frac{1}{x}\Big)^x=e\)&
        \(\displaystyle \lim_{x\to \infty}\Big(1+\frac{k}{x}\Big)^x=e^k\)\\
        \(\displaystyle \lim_{x\to 0}\frac{a^x-1}{x}=\textrm{ln}{a}\)&
        \(\displaystyle \lim_{x\to 0}\frac{\textrm{ln}(1\pm x)}{x}=\pm 1\)\\
      \end{tabular}
    \end{center}
   \newpage





\section{Successioni}
  \subsection{Definizioni}
    \begin{enumerate}
      \item una successione \'e una funzione $a_{n}: \mathbf{N} \rightarrow \mathbf{R}$
      \item un insieme $K\subset \mathbf{R}$ si dice compatto se da ogni successione a valori in $K$
        si può estrarre una sottosuccessione convergente a un punto di $K$
      \item una successione è definita monotona crescente se $\forall n\subset \mathbf{N}:
        a_{n+1}\ge a_{n}$
        mentre monotone decrescente se $\forall n\subset \mathbf{N}: a_{n+1}\le a_{n}$ 
      \item Una successione si dice di Cauchy se $\forall \varepsilon>0 \exists \nu:
        \forall n,m>\nu \quad |a_{m}-a_{n}|<\varepsilon$  
     \end{enumerate}
  
  
  
  \subsection{Teoremi}
    \begin{enumerate}
      \item Da ogni successione limitata si può estrarre una sottosuccessione convergente
      \item una successione $a_{n}$ monotona ha sempre limite, se $a_{n}$ \'e crescente si ha 
      \(\displaystyle \lim_{n\to \infty}a_{n}= \sup_{n\in \mathbf{N}}a_{n}\),
      mentre se $a_{n}$ \'e decrescente
      \(\displaystyle \lim_{n\to \infty}a_{n}= \inf_{n\in \mathbf{N}}a_{n}\),
    \end{enumerate}
  \newpage





\section{Serie}
 \subsection{Definizioni}
 
 
 
 \subsection{Teoremi}
  \newpage





\section{Funzioni continue}
  \subsection{Teoremi}
    \begin{enumerate}
      \item
        siano $f(x)$ una funzione continua in $x_{0}$ e
        $g(y)$ una funzione continua in $y_{0}=f(x_{0})$, allora
        la funzione composta $g(f(x))$ \'e continua in $x_{0}$
        \begin{displaymath}
          x_{0}\cup y_{0}\subset \mathbf{R}:f(x) \in x_{0} \land g(y) 
          \in y_{0}=f(x_{0}) \Rightarrow f(g(x))\in x_{0}
        \end{displaymath}
      \item
        se $g(x)$ \'e una funzione continua, allora
        \begin{displaymath}
          g:A\rightarrow \mathbf{R}\Rightarrow\lim_{x\to x_{0}}g(f(x))=g(\lim_{x\to x_{0}}f(x))
        \end{displaymath}
    \end{enumerate}
\end{document}