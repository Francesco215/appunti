\documentclass{exam}
\date{Marzo 2017}
\usepackage[italian]{babel}
\usepackage[T1]{fontenc}
\usepackage{mathrsfs}
\title{Appunti di Analisi Matematica}
\author{Francesco Sacco}
\begin{document}





\section{Limiti}
  \subsection{Definizioni}
    \begin{enumerate}
      \item 
        sia $f(x) \in D\subset R$ e sia $x_{0}$ un punto di accumulazione di $D$. diremo che
        \begin{displaymath}
          \lim_{x \to x_{0}}=L\Leftrightarrow \forall \varepsilon , \exists \delta > 0 :
          \forall x \in D \land 0<|x-x_{0}|< \delta) \Rightarrow |f(x)-L|<\varepsilon
        \end{displaymath}
      \item
        $\textrm{sia } f(x) \in D \subset \mathbf{R} \land x_{0}\in f(x) $
        \begin{displaymath}
          \textrm{se } \forall M \in \mathbf{R} \exists \delta >0: \forall x \subset D , 
          0<|x-x_{0}|<\delta \Rightarrow f(x)>M
          \textrm{ allora} \lim_{x \to x_{0}}f(x)=\infty
        \end{displaymath}
      \item 
        sia $f(x) \in D \subset \mathbf{R}$ non limitato superiormente
        \begin{displaymath}
          \lim_{x \to \infty}=L \Leftrightarrow 
          \forall \varepsilon>0 \exists \nu:\forall x\in D,
          x>\nu \Rightarrow |f(x)-L|<\varepsilon
        \end{displaymath}
      \item
        \(\displaystyle
          \textrm{sia } f(x) \in D\subset \mathbf{R}
          \textrm{ se }\lim_{x\to x_{0}}=\infty \textrm{ allora diremo che la funzione }f(x) 
          \textrm{ ha un'asintoto verticale in }x_{0}
        \)
      \item 
        sia $f(x)\in (c, \pm \infty)$ Diremo che la retta di equazione $y=ax+b$
        \'e un asintoto di $f$ se
          \begin{displaymath}
            \lim_{x\to \pm \infty}[f(x)-ax-b]=0
          \end{displaymath}
    \end{enumerate}
  
  
  \subsection{Teoremi}
    \begin{enumerate}
      \item siamo $f(x)$ e $g(x)$ due funzioni tali che
        \begin{displaymath}
          \lim_{x\to x_{0}}f(x)=L
        \end{displaymath}
        \begin{displaymath}
          \lim_{x\to x_{0}}g(x)=M
        \end{displaymath}
        \begin{displaymath}
          \textrm{allora}
        \end{displaymath}
        \begin{displaymath}
          \lim_{x \to x_{0}} [f(x)+g(x)]= L+M
        \end{displaymath}
        \begin{displaymath}
          \lim_{x \to x_{0}} f(x)g(x)=LM
        \end{displaymath}
        \begin{displaymath}
          \lim_{x\to x_{0}} \frac{f(x)}{g(x)}=\frac{L}{M}
        \end{displaymath}
      \item{teorema della permanenza del segno}
        \begin{displaymath}
          \textrm{se} \lim_{x\to x_{0}} f(x)=L>0 \Rightarrow \exists \delta >0 : \forall x :
          0<|x-x_{0}|<\delta \textrm{ si ha } f(x)>\frac{M}{2}>0
        \end{displaymath}
      \item{teorema dei 2 carabinieri}
        \begin{displaymath}
          \textrm{siano} f(x),g(x)\textrm{ e } h(x) \in \mathbf{R} \land f(x)<g(x)<h(x) 
          \land \lim_{x \to x_{0}}f(x)=\lim_{x \to x_{0}}h(x)=L
        \end{displaymath}
        \begin{displaymath}
          \textrm{allora }\lim_{x \to x_{0}}g(x)=L
        \end{displaymath}
    \end{enumerate}
  
  
  \subsection{Limiti notevoli}
    \begin{center}
      \begin{tabular}{|c|c|}
        \(\displaystyle \lim_{x\to 0}\frac{\textrm{sen}(x)}{x}=1\)&
        \(\displaystyle \lim_{x\to 0}\frac{1-\textrm{cos}(x)}{x^2}=\frac{1}{2}\)\\
        \(\displaystyle \lim_{x\to 0}\frac{\textrm{tan}(x)}{x}=1\)&
        \(\displaystyle \lim_{x\to 0}\frac{\textrm{arcCos}(x)}{\sqrt[]{1-x}}=\sqrt[]{2}\)\\
        \(\displaystyle \lim_{x\to 0}(1+x)^\frac{1}{x}=e\)&
        \(\displaystyle \lim_{x\to 0}(1-x)^\frac{1}{x}=\frac{1}{e}\)\\
        \(\displaystyle \lim_{x\to \infty}\Big(1+\frac{1}{x}\Big)^x=e\)&
        \(\displaystyle \lim_{x\to \infty}\Big(1+\frac{k}{x}\Big)^x=e^k\)\\
        \(\displaystyle \lim_{x\to 0}\frac{a^x-1}{x}=\textrm{ln}{a}\)&
        \(\displaystyle \lim_{x\to 0}\frac{\textrm{ln}(1\pm x)}{x}=\pm 1\)\\
      \end{tabular}
    \end{center}
   \newpage





\section{Successioni}
  \subsection{Definizioni}
    \begin{enumerate}
      \item una successione \'e una funzione $a_{n}: \mathbf{N} \rightarrow \mathbf{R}$
      \item un insieme $K\subset \mathbf{R}$ si dice compatto se da ogni successione a valori in $K$
        si pu\'o estrarre una sottosuccessione convergente a un punto di $K$
      \item una successione \'e definita monotona crescente se $\forall n\subset \mathbf{N}:
        a_{n+1}\ge a_{n}$
        mentre monotone decrescente se $\forall n\subset \mathbf{N}: a_{n+1}\le a_{n}$ 
      \item Una successione si dice di Cauchy se $\forall \varepsilon>0 \exists \nu:
        \forall n,m>\nu \quad |a_{m}-a_{n}|<\varepsilon$  
     \end{enumerate}
  
  
  \subsection{Teoremi}
    \begin{enumerate}
      \item Da ogni successione limitata si pu\'o estrarre una sottosuccessione convergente
      \item una successione $a_{n}$ monotona ha sempre limite, se $a_{n}$ \'e crescente si ha 
      \(\displaystyle \lim_{n\to \infty}a_{n}= \sup_{n\in \mathbf{N}}a_{n}\),
      mentre se $a_{n}$ \'e decrescente
      \(\displaystyle \lim_{n\to \infty}a_{n}= \inf_{n\in \mathbf{N}}a_{n}\),
    \end{enumerate}





\section{Serie}
 \subsection{Definizioni}
  \begin{enumerate}
    \item
      sia $a_{i}+a_{i+1}+\dots+a_{j}$ la somma dei terimini tra $i$ e $j$ della successione $a_{n}$ essa \'e definita come \(\displaystyle \sum_{n=i}^j a_{n}\)
    \item
      se \(\displaystyle \forall n\in \mathbf{R}\,|\, s_{n}=\sum_{k=0}^n a_{k}\), allora $s_{n}$ \'e definita la successione delle somme parziali
  \end{enumerate}
 
 
 \subsection{Teoremi}
  \begin{enumerate}
    \item
      sia $\sum a_{n}$ una serie a termini positivi, e sia $s_{n}$ la successione delle somme parziali. Se la successione $s_{n}$ \'e limitata superiormente, la serie converge, altrimenti diverge a $+\infty$
    \item
      se serie $\sum a_{n}$ \'e convergente, allora la successione $a_{n}$ \'e infinitesima
    \item{criterio del confronto}\newline
      siano $\sum a_{k}$ e $\sum b_{k}$, e supponiamo che $\forall n\in \mathbf{R}\,|\, 0\le
      a_{n}\le b_{n}$, allora se $\sum b_{k}$ converge, allora $\sum a_{k}$ converge, viceversa se $\sum a_{k}$ diverge, allora anche $\sum b_{k}$ diverge
    \item{criterio del confronto asintotico}\newline
      siano $a_{n}$ e $b_{n}$ serie a termini positivi
      \begin{displaymath} 
        \sum_{n}^\infty b_{n} \textrm{ converge}\land \lim_{n\to \infty}\frac{a_{n}}{b_{n}}=L\in \mathbf{R} \Rightarrow \sum_{n}^\infty a_{n} \textrm{ converge}
      \end{displaymath}
    \item{criterio della radice}\newline
      sia $\sum a_{n}$ una serie a termini positivi, se \(\displaystyle \lim_{n\to \infty}\sqrt[n]{a_{n}}=L<1\), allora la serie converge
    \item{criterio dell'assoluta convergenza}\\
      se $\sum |a_{n}|$ converge, allora $\sum a_{n}$ converge anch'essa
    \item
      sia $a_{n}$ una serie a termini positivi, se
      \begin{displaymath}
        \lim_{n\to \infty}a_{n}=0\Rightarrow \sum {(-1)}^{n}a_{n} \textrm{ converge}
      \end{displaymath}
  \end{enumerate}
  \newpage





\section{Funzioni continue}
  \subsection{Definizioni}
    \begin{enumerate}
      \item
        una funzione si dice continua se \(\displaystyle \lim_{x\to x_{0}}f(x)=f(x_{0})\)
      \item
        sia $f(x)$ una funzione definita in un insieme $A$, e sia $x_{0}$ un punto di $A$. Diremo che $x_{0}$ \'e un punto di massimo assoluto se risulta che $f(x)\le f(x_{0}) \forall x\in A$
      \item
        sia $f(x)$ una funzione definita in un insieme $A$, e sia $x_{0}$ un punto di $A$. Diremo che $x_{0}$ \'e un punto di massimo relativo se in un'intorno $I\subset A\,|\,f(x)\le f(x_{0}) \forall x\in I$
    \end{enumerate}


  \subsection{Teoremi}
    \begin{enumerate}
      \item
        siano $f(x)$ una funzione continua in $x_{0}$ e
        $g(y)$ una funzione continua in $y_{0}=f(x_{0})$, allora
        la funzione composta $g(f(x))$ \'e continua in $x_{0}$
        \begin{equation}
          x_{0}\cup y_{0}\subset \mathbf{R}:f(x) \in x_{0} \land g(y) 
          \in y_{0}=f(x_{0}) \Rightarrow f(g(x))\in x_{0}
        \end{equation}
      \item
        se $g(x)$ \'e una funzione continua, allora
        \begin{equation}
          g:A\rightarrow \mathbf{R}:\lim_{x\to x_{0}}g(f(x))=g(\lim_{x\to x_{0}}f(x))
        \end{equation}
      \item
        una funzione $f:A\rightarrow \mathbf{R}$ \'e continua se e solo se per ogni successione $x_{n}$ a valori in $A$ convergente a $x_{0}$, la successione $f(x_{n})$ converge a $f(x_{0})$
      \item{teorema della permanenza del segno}\\
        sia $f(x)$ una funzione continua in $A$, e sia $x_{0}$ un punto di $A$. Se risulta $f(x_{0})>0$, allora esiste un'intordo $I$ di $x_{0}$ tale che per ogni $x\in I\cap A$ si ha $f(x)>0$
      \item{teorema degli zeri delle funzioni continue}\\
        sia $f(x)$ una funzione continua in un intervallo $[a,b]$, se $f(a)>0 \lor f(b)<0$, allora esiste un punto $x_{0}\in (a,b):f(x_{0})=0$
      \item{teorema dei valori intermedi}\\
        una funzione $f(x)$ continua in un intervallo $I$ assume tutti i valori compresi tra $\inf_{I}f$ e $\sup_{I}f$
      \item{teorema della continuit\'a della funzione inversa}\\
        sia $f(x)$ una funzione continua e invertibile in un intervallo $I$ che pu\'o essere una semiretta o tutto $\mathbf{R}$. Allora la sua inversa \'e continua
      \item le funzioni goniometriche e le loro inverse sono continue
      \item{Teorema di Weistrass}\\
        Una funzione continua in un insieme $E$ compatto ha massimo e minimo
    \end{enumerate}
  \newpage





\section{Uniforme continuit\'a}
  \subsection{Definizoni}
    \begin{enumerate}
      \item
        Una funzione \'e uniformemente continua se
        \begin{equation}
         \forall \epsilon >0, \exists \delta >0 : \forall x_{0},x_{1} \in Dom(f),[|x_{0}-x{1}|<\delta \Rightarrow |f(x_{0})-f(x_{1})|<\epsilon]
        \end{equation}
      \item
       una funzione $f(x)$ definita in $A$ \'e detta lipsichitziana se $ \forall x\in A,\exists L\in \mathbf{R}:|f'(x)|\le L$
  \end{enumerate}


  \subsection{Teoremi}
    \begin{enumerate}
      \item
        sia $f:A\rightarrow \mathbf{R}$ se essa \'e lipsichitiziana in A, allora \'e anche uniformemente continua
      \item{Teorema di Heine-Cantor}\\
        se $f(x)$ \'e definita in un inzieme chiuso e limitato, allora \'e uniformemente continua
      \item{Osservazione di Heine-Cantor}\\
        se $f(x)$ \'e continua in $[a,b]$, allora \'e uniformemente continua
      \item
        sia $f:[a,+\infty)\rightarrow \mathbf{R}$ continua, se la funzione ha un asintoto non verticale, allora da un certo punto in poi \'e uniformemente continua
      \item
        se $f(x)$ \'e un. continua in $[a,b)$ e $(b,c]$, allora essa \'e uniformemente continua in $[a,c]$
      \item
        siano $f,g:A\rightarrow \mathbf{R}$ funzioni un. continue, allora $f+g$ \'e un. continua
      \item
        siano $f:A\rightarrow B$ e $g:B\rightarrow \mathbf{R}$ funzioni un. continue, allora $g[f(x)]$ \'e un. continua
    \end{enumerate}    
  \newpage






\section{Derivata}
  \subsection{Definizioni}
    \begin{enumerate}
      \item
        sia $f:(a,b)\rightarrow \mathbf{R}$ e sia $x$ un punto di $(a,b)$. Diremo che $f$ \'e derivabile in $x$ se esiste il limite finito
        \begin{equation}
          f'(x)=\lim_{h\to 0}\frac{f(x+h)-f(x)}{h} 
        \end{equation}
    \end{enumerate}


  \subsection{Teoremi}
    \begin{enumerate}
      \item
        se una funzione $f$ \'e derivabile in un punto $x_{0}$, \'e continua in $x_{0}$
      \item 
        sia $f(x)$ una funzione definita in $A\subset \mathbf{R}$, e sia $x_{0}\in A$ un punto stazionario. se $f$ \'e derivabile in $A$, allora $f'(x_{0})=0$
      \item{Teorema di Rolle}\\
        sia $f(x)$ una funzione continua in un'intervallo chiuso $[a,b]$, derivabile in $(a,b)$ e tale che $f(a)=f(b)$. Allora esiste almeno un punto compreso tra $a$ e $b$ in cui la derivata si annulla
      \item{Teorema di Lagrange}\\
        sia $f(x)$ una funzione continua in un'intervallo chiuso $[a,b]$, derivabile in $(a,b)$. esiste un punto $\xi \in (a,b)$ tale che
        \begin{equation}
          f'(\xi)=\frac{f(b)-f(a)}{b-a}
        \end{equation}
      \item{Teorema di Caucy}\\
        siano $f(x)$ e $g(x)$ 2 funzioni continue in un'intervallo chiuso $[a,b]$ e derivabili in $(a,b)$, allora esiste un punto $\xi$ tale che
        \begin{equation}
          \frac{f'(\xi)}{g'(\xi)}=\frac{f(b)-f(a)}{g(b)-g(a)}
        \end{equation}
      \item
        sia $f(x)$ una funzione derivabile definita in un intervalo $I$. Se $\forall x\in I\,|\,f'(x)=0$, allora $f(x)$ \'e costante
      \item
        una funzione $f(x)$ derivabile in un'intervallo $I$ \'e crescente se e solo se $f'(x)\ge 0$
      \item{Teorema di de l'Hopital}\\
        siano $f(x)$ e $g(x)$ funzioni derivabili in un'intervallo $I$, con la possibile eccezzione di $x_{0}\in I$, supponiamo che $f(x_{0})=g(x_{0})=0$, e supponiamo che esista il limite del rapporto delle derivate, allora
        \begin{equation}
          \lim_{x\to x_{0}}\frac{f(x)}{g(x)}=\lim_{x\to x_{0}}\frac{f'(x)}{g'(x)}=L
        \end{equation}
    \end{enumerate}
  \newpage





\section{Integrale}
	\subsection{Definizioni}
		\begin{enumerate}
			\item{Funzione costante a tratti}
				sia $\varphi(x)_{I_{k}}$ una funzione che vale $1$ nell'intervallo $I_{k}$ e zero in tutto il resto e sia $\lambda_{k}$ una costante, allora la funzione $\varphi(x)=\lambda_{1}\varphi_{I_{1}}(x)+\lambda_{2}\varphi_{I_{2}}(x)+\dots+\lambda_{n}\varphi_{I_{n}}(x)=\sum\lambda_{k}\varphi_{I_{k}}$ \'e definita costante a tratti
			\item{Integrale funzione semplice}\\
				sia $\varphi(x)=\sum\lambda_{k}\varphi_{I_{k}}$ una funzione costante a tratti, siano $I_1 \cup I_2 \cup \dots \cup I_n \subseteq [a,b)$,  siano $x_{k}$ e $x_{k}'$ gli estremi dell'insieme $I_{k}$ , allora 
				\begin{equation}
					\int_a^b \varphi(x)=\sum \lambda_{k}(x_{k}-x_{k}')
				\end{equation}
			\item{Integrale definito di Reimann}\\
				sia f(x) una funzione limitata definita nell'intervallo $I=[a,b)$. Indichiamo con $\mathscr{S}^+$ la classe delle funzioni semplici $\psi$ maggioranti tali che $\psi(x)\geq f(x)$ in $[a,b)$ e con $\mathscr{S}^-$ la classe delle funzioni semplici $\varphi$ minoranti tali che $\varphi(x)\le f(x)$ in $[a,b)$ e siano $\psi(x)=\varphi(x)=0$ se $x$ non appartiene a $[a,b)$, allora se si ha che 
				\begin{equation}
					\sup_{\psi\in\mathscr{S}^+}\int_a^b \psi(x)dx=\inf_{\varphi\in\mathscr{S}^-}\int_a^b \varphi(x)dx
				\end{equation}
				il loro valore comune sar\'a uguale all'integrale in $[a,b)$ di $f(x)$ che verr\'a indicata col simbolo
				\begin{equation}
					\int_a^b f(x)dx
				\end{equation}
			\item{Integrale indefinito}\\
				sia $f(t)$ una funzione integrabile, allora la funzione \(\displaystyle F(x)=\int_0^x f(t)dt\) \'e definita il suo integrale indefinito e verr\'a indicato cos\'i
				\begin{equation}
					\int f(x)dx
				\end{equation}
			\item{Integrale di Reimann generalizzato 1}\\
				sia $f(x)$ integrabile in $(a,b]$ e non limitata in $a$, e sia $c\in (a,b]$, allora 
				\begin{equation}
					\int_a^b f(x)dx = \lim_{c\to a}\int_c^b f(x)dx
				\end{equation}
			\item{Integrale di Reimann generalizzato 2}\\
				sia $f(x)$ integrabile in $[a,+\infty)$, allora
				\begin{equation}
					\int_a^\infty f(x)dx=\lim_{b\to \infty}\int_a^b f(x)dx
				\end{equation}
			\item
				un funzione $f$ viene detta assolutamente integrabile in $(a,+\infty)$ se esiste un $k\in\mathbf{R}$ tale che 
				\begin{equation}
					\int_a^{+\infty}|f(x)|dx < k
				\end{equation}
		\end{enumerate}
			




	\subsection{Teoremi}
		\begin{enumerate}
			\item
				se $f(x)$ \'e continua in $[a,b]$, allora \'e integrabile in $[a,b)$
			\item
				se $f(x)$ \'e continua e limitata in $(a,b]$, essa \'e integrabile in $[a,b]$
			\item
				se $f(x)$ \'e crescente in $[a,b]$ allora \'e integrabile
			\item 
				se $f(x)$ \'e integrabile, $|f(x)|$ \'e integrabile
			\item{Teorema della media integrale}\\
				sia $f(x)$ integrabile in $I=[x_1,x_2]$ allora 
				\begin{equation}
					\inf_{x\in I}f(x)\le\frac{1}{x_1-x_2}\int_{x_1}^{x_2} f(x)dx\le \sup_{x\in I}f(x)
				\end{equation}
			\item{Teorema dei valori intermedi}\\
				sia $f(x)$ una funzione integrabile in $[x_1,x_2]$, esiste un punto $\xi$ compreso tra $x_1$ e $x_2$ tale che 
				\begin{equation}
					f(\xi)=\frac{1}{x_1-x_2}\int_{x_1}^{x_2} f(x)dx
				\end{equation}
			\item{Teorema fondamentale del calcolo integrale}\\
				sia $f$ una funzione integrabile in $[a,b]$ e sia \(\displaystyle F(x)=\int_a^x f(t)dt \), allora $F'(x)=f(x)$.
				Sia $G(x)$ una funzione tale che $G'(x)=f(x)$, allora $F'(x)=G(x)-G(a)$
			\item
				sia $f$ una funzione integrabile in $[a,b]$, allora \(\displaystyle F(x)=\int_a^x f(t)dt \) \'e lispsichitiziana
			\item
				sia $f$ una funzione assolutamente integrabile in $(a,b)$, allora $f$ \'e anche integrabile e in particolare
				\begin{equation}
					\bigg|\int_a^b f(x)dx\bigg|\le \int_a^b|f(x)|dx
				\end{equation}
			\item
				sia $f$ una funzione continua integrabile in $(a,+\infty)$, allora se \(\displaystyle \int_a^{+\infty}f(x)dx=L\in\mathbf{R}\Rightarrow \lim_{x\to\infty}f(x)=0\) 
			\item
				sia $f:[a,+\infty)\rightarrow\mathbf{R}$ e $a_n$ una sucessione con $n\geq a$ se $f(n)=a_n$ per ogni $n\geq a$, allora
				\begin{equation}
					\int_a^\infty f(x)dx	\le \sum_n^\infty a_n \le f(a)+ \int_a^\infty f(x)dx 
				\end{equation}
			\item{criterio dell'integrale per le serie}\\
				sia $f:[a,+\infty)\rightarrow\mathbf{R}$ e $a_n$ una sucessione con $n\geq a$ se $f(n)=a_n$ per ogni $n\geq a$, allora se
				\begin{equation}
					\sum_n^\infty a_n \textrm{ converge}\Leftrightarrow \int_a^\infty f(x)dx \textrm{ converge}
				\end{equation}

		\end{enumerate}







\end{document}