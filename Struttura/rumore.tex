\documentclass[10pt,a4paper]{article}
\usepackage{verbatim}
\usepackage{subcaption}
\usepackage{karnaugh-map}
\usepackage[utf8]{inputenc}
\usepackage[italian]{babel}
\usepackage{amsmath}
\usepackage{amsfonts}
\usepackage{amssymb}
\usepackage{graphicx}
\usepackage{mathrsfs}
\usepackage[left=2cm,right=2cm,top=2cm,bottom=2cm]{geometry}
\newcommand{\rem}[1]{[\emph{#1}]}
\newcommand{\exn}{\phantom{xxx}}
\newcommand{\limx}{\lim_{dx\to 0}}

\author{Francesco Sacco}
\title{Fluttuazioni e Rumore}
\begin{document}
\date{Agosto 2018}
\maketitle

\section{Fluttuazioni numero di Particelle}
	Il potenziale di landau $\Omega=E-TS-\mu N$, quindi $d\Omega=-SdT-PdV-Nd\mu$ ne consegue che $\partial \Omega/\partial \mu=-N$. Questa relazione può essere dimostrata attraverso la seguente relazione.

	\begin{equation}
		\Omega=-kT\ln Z\exn|\exn Z=\sum_{stati}\exp\bigg(\frac{\mu N -E}{kT}\bigg)
	\end{equation}

	Per $\sum_{stati}$ si intende dire di sommare su tutti i possibili numeri $N$ di particelle e su tutte le possibili configurazioni di posizioni e impulsi di quelle $N$ particelle.\newline

	Prima di vedere quanto vale quella somma calcoliamoci per sport $\partial \Omega/\partial \mu$ e $\partial^2 \Omega/\partial \mu^2$.

	\[
		\frac{\partial \Omega}{\partial \mu}=\frac{-kT}{Z}\frac{\partial}{\partial \mu}
		\bigg[\sum_{stati}\exp\frac{\mu N -E}{kT}\bigg]=-\frac1Z \sum_{stati}N\exp\frac{\mu N -E}{kT}=-\bar N
	\]\newline\newline
	\[
		\frac{\partial^2 \Omega}{\partial \mu^2}=
		-\frac{\partial}{\partial \mu}\bigg[\frac1Z \sum_{stati}N\exp\frac{\mu N -E}{kT}\bigg]=
	\]
	\[
		=\frac1{kTZ^2}\bigg[\sum_{stati}N\exp\frac{\mu N -E}{kT}\bigg]^2-\frac1{kTZ}\sum_{stati}N^2\exp\frac{\mu N -E}{kT}=\frac1{kT}\bigg[\big(\bar N\big)^2-\bar{\big(N^2\big)}\bigg]=-\frac{\big(\Delta N\big)^2}{kT}
	\]
	Quindi alla fine si ha che:
	\begin{equation}
		-kT\frac{\partial^2 \Omega}{\partial \mu^2}=\big(\Delta N\big)^2
	\end{equation}


\section{Potenziali di Landau per Fermi e Bose}

	\Omega_q=-Kt\ln



\end{document}