\documentclass[10pt,twoside]{article}

\usepackage{blindtext} % Package to generate dummy text throughout this template 

\usepackage[sc]{mathpazo} % Use the Palatino font
\usepackage[T1]{fontenc} % Use 8-bit encoding that has 256 glyphs
\linespread{1.05} % Line spacing - Palatino needs more space between lines
\usepackage{microtype} % Slightly tweak font spacing for aesthetics
\usepackage{graphicx}
\usepackage{multirow}
\setlength\parindent{0pt}
\usepackage[english]{babel} % Language hyphenation and typographical rules
\usepackage[hmarginratio=1:1,top=32mm,columnsep=20pt, total={6.8in, 9in}]{geometry} % Increase total to increase margins 
\usepackage[hang, small,labelfont=bf,up,textfont=it,up]{caption} % Custom captions under/above floats in tables or figures
\usepackage{booktabs} % Horizontal rules in tables

\usepackage{braket}

\usepackage[export]{adjustbox} % valign package

\usepackage{lettrine} % The lettrine is the first enlarged letter at the beginning of the text

\usepackage{enumitem} % Customized lists
\setlist[itemize]{noitemsep} % Make itemize lists more compact

\usepackage{abstract} % Allows abstract customization
\renewcommand{\abstractnamefont}{\normalfont\bfseries} % Set the "Abstract" text to bold
\renewcommand{\abstracttextfont}{\normalfont\itshape\normalsize} % Set the abstract itself to small italic text

\usepackage{titlesec} % Allows customization of titles
\renewcommand\thesection{\Roman{section}} % Roman numerals for the sections
\renewcommand\thesubsection{\roman{subsection}} % roman numerals for subsections
\titleformat{\section}[block]{\large\scshape\centering}{\thesection.}{1em}{} % Change the look of the section titles
\titleformat{\subsection}[block]{\large}{\thesubsection.}{1em}{} % Change the look of the section titles

\usepackage{fancyhdr} % Headers and footers
\pagestyle{fancy} % All pages have headers and footers
\fancyhead{} % Blank out the default header
\fancyfoot{} % Blank out the default footer
\fancyhead[C]{University of Pisa} % Custom header text
\fancyfoot[RO,LE]{\thepage} % Custom footer text
\pagenumbering{gobble}
\usepackage{titling} % Customizing the title section

\usepackage{hyperref} % For hyperlinks in the PDF
\usepackage[ruled,vlined]{algorithm2e}

\usepackage[backend=bibtex]{biblatex}
\addbibresource{bib_source.bib}

\usepackage{multicol,caption}

\newenvironment{Figure}
  {\par\medskip\noindent\minipage{\linewidth}}
  {\endminipage\par\medskip}

\newenvironment{Algorithm}[1][htb]
  {\renewcommand{\algorithmcfname}{Algorithm}% Update algorithm name
   \begin{algorithm}[#1]%
  }{\end{algorithm}}


\setlength{\droptitle}{-4\baselineskip} % Move the title up

\pretitle{\centering \Huge\bfseries} % Article title formatting
%\posttitle % Article title closing formatting
\title{Why the Universe can't be Infinite and why it Expands} % Article title
\author{%
\textsc{Francesco Sacco}\\[1ex] % Second author's name
\normalsize \href{mailto:francesco215@live.it}{francesco215@live.it} % Second author's email address
}
\date{
    \small\today\\
    \small{\small{University of Pisa}}} % Leave empty to omit a date



\begin{document}

\maketitle

\begin{abstract}
The maximum entropy a region of space can cointain is proportional to the surface area
\end{abstract}
\vspace{0.2cm}
\begin{multicols}{2}

\section{Why the universe can't be Infinite}
	The entropy of a black hole is
	\begin{equation}
		S=\frac{k_b c^3}{4G\hbar} A=\alpha V^{2/3}
	\end{equation}
	where
	\begin{equation}
		\alpha=\frac{k_b c^3}{4G\hbar} \sqrt[3]{3}\bigg(\frac34\bigg)^{2/3}
	\end{equation}

	and $A$ and $V$ is the surface of the black hole\newline
	Therefore the entropy per unit volume of a black hole is
	\begin{equation}
		s_b=\frac SV=\frac{\alpha}{\sqrt[3]{V}}
	\end{equation}
	And it decreases as the volume of the black hole gets larger and larger.


	The main property of black holes that we are going to keep in mind is that a black hole of volume $V$ has the maximum entropy any spherical volume $V$ can have.


	Suppose we have a universe enclosed in a volume of radius $R$ and that on large scales we can say that it has constant entropy density $s_u=dS/dV$.\newline
	The maximum volume of this universe $V_{max}$ is
	\begin{equation}
		V_{max}= \bigg(\frac\alpha {s_u}\bigg)^3
	\end{equation}

	Therefore the univere can't be infinite if it is homogeneus on large scales and flat
\section{Why the universe expands}
	Suppose the volume of the Universe for some reason is exactly $V_{max}$ it would last a blink of an eye before collapsing into a black hole. That is because the total entropy has to increase.
	If the universe has some kind of law that keeps it's volume always equal to $V_{max}$ we would have an expanding universe.

	This could explain why our universe in expanding, and why the expantion of the universe isn't constant throught the age of the universe.

\end{multicols}
\end{document}