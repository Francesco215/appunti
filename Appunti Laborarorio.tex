\documentclass{exam}
\date{Dicembre 2016}
\usepackage[italian]{babel}
\usepackage[T1]{fontenc}
\title{Appunti di Laboratorio}
\author{Francesco Sacco}
\usepackage{amsmath}
\begin{document}


  \section{Teoria delle probabilit\'a}
    \subsection{Coeafficenti binomiali}
      dato un insieme di $n$ elementi \'e possibile ordirarli in 
      $\binom{n}{k}$ gruppi di dimenzione $k$
      \[ \binom{n}{k}=\frac{n!}{k!(n-k)!} \]
      inoltre la potenza $n$-esima di un binomio \'e uguale a 
      \(\displaystyle(a+b)^n=\sum_{k=0}^n \binom{n}{k}a^k b^{k-n} \)
      
    \subsection{Teorema di Bayes}
      \[P(E_{1}|E_{2})=\frac{P(E_{2}|E_{1})P(E_{1})}{P(E_{2})} \]
      
    \subsection{Media ($\mu$)}
      \begin{displaymath}
        \mu=\textrm{E}[x]=\begin{cases}
            \displaystyle \int_{-\infty}^{+\infty}xp(x)dx \textrm{ per variabili continue}\\
            \displaystyle \sum_{k=0}^{n}x_{k}\textrm{P}(x_{k})
            \textrm{ per variabili discrete}
          \end{cases}
      \end{displaymath}
      E$[x]$ \'e un'operatore lineare quindi E$[ax]=$E$a[x]$
       
     \subsection{Varianza $(\sigma^2)$}
       \[
         \sigma^2=\textrm{E}\bigr[(x-\mu)^2\bigr]=\begin{cases}
             \displaystyle \int_{-\infty}^{+\infty}(x-\mu)^p(x)dx 
             \textrm{ per variabili continue}\\
             \displaystyle \sum_{k=0}^{n}(x_{k}-\mu)^2\textrm{P}(x-{k})
             \textrm { per variabili continue}
           \end{cases}
       \]
\end{document}