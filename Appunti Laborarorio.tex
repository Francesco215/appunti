\documentclass{exam}
\date{Dicembre 2016}
\usepackage[italian]{babel}
\usepackage[T1]{fontenc}
\title{Appunti di Laboratorio}
\author{Francesco Sacco}
\usepackage{amsmath}
\begin{document}


  \section{Basi}
    \subsection{Coeafficenti binomiali}
      dato un insieme di $n$ elementi \'e possibile ordirarli in 
      $\binom{n}{k}$ gruppi di dimenzione $k$
      \[ \binom{n}{k}=\frac{n!}{k!(n-k)!} \]
      inoltre la potenza $n$-esima di un binomio \'e uguale a 
      \(\displaystyle(a+b)^n=\sum_{k=0}^n \binom{n}{k}a^k b^{k-n} \)
      
    \subsection{Teorema di Bayes}
      \[P(E_{1}|E_{2})=\frac{P(E_{2}|E_{1})P(E_{1})}{P(E_{2})} \]
      
    \subsection{Media ($\mu$)}
      \begin{displaymath}
        \mu=\textrm{E}[x]=\begin{cases}
            \displaystyle \int_{-\infty}^{+\infty}xp(x)dx \textrm{ per variabili continue}\\
            \displaystyle \sum_{k=0}^{n}x_{k}\textrm{P}(x_{k})
            \textrm{ per variabili discrete}
          \end{cases}
      \end{displaymath}
      E$[x]$ \'e un'operatore lineare quindi E$[ax]=$E$a[x]$
      e se $x_{1} \lor x_{2}$ sono variabili indipendenti, allora
      E$[x_{1}x_{2}]$=E$[x_{1}]$E$[x_{2}]$=$\mu_{1}\mu_{2}$
      Il calcolo della media resta identica anche nel caso in cui si effettuano delle misurazioni, infatti \(\displaystyle m=\frac{1}{n}\sum_{i=1}^{n}x_{i}\)
       
    \subsection{Varianza $(\sigma^2)$}
      \[
        \sigma^2=\textrm{E}\Big[(x-\mu)^2\Big]=\begin{cases}
            \displaystyle \int_{-\infty}^{+\infty}(x-\mu)^2 p(x)dx 
            \textrm{ per variabili continue}\\
            \displaystyle \sum_{k=0}^{n}(x_{k}-\mu)^2\textrm{P}(x_{k})
            \textrm { per variabili continue}
          \end{cases}
      \]
      Var$[cx]=c^2$Var$[x]$ e se $x_{1} \lor x_{2}$ sono variabili indipendenti, allora
      $\sigma^2=\sigma_{1}^2+\sigma_{2}^2$

      Precedentemente abbiamo analizzato il caso in cui si conosce la funzione di distribuzione, nel caso in cui gli unici dati a disposizione sono delle misurazioni, allora $\sigma^2$ viene stimata nel seguente modo
      \begin{displaymath}
        \textrm{E}\Big[s_{n-1}^2\Big]=
        \textrm{E}\Bigg[\frac{1}{n-1}\sum_{i=1}^{n}(x_{i}-m)^2\Bigg]=\sigma^2 
      \end{displaymath}
  \section{Ora si f\'a sul serio}
\end{document}